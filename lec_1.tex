\documentclass{article}
\usepackage{amsmath,latexsym,babel,amssymb,amsfonts,geometry,physics}
\usepackage[dvipsnames]{xcolor}

\title{QM2 }
\author{Pushkar Mohile}
\newcommand{\note}[1]{\color{blue} Note: #1}
\begin{document}
 \maketitle 
 \section{Lec 1}
 \subsection{Stuff from QM1}
 The course is very important. More important than QM1. QM1 teaches special problems. Real quantum system 
 Need approximation methods. 
 QM - linear vector space. 
 SHO, ladder operator Solve directly or use ladder operators. Clebsch-Gordon, Addition of Angular momentum. 
We study appproximation methods. Perturbation theory. \\
Eg: He-4 cannot be solved exactly using the Schrodinger Equation. But variational methods are good for this. 
 Require an approximate numerical solution.
 Scattering: You scatter particles from a potential scattering center (Rutherford scattering)
 \subsection{Review}
 Ladder operators are generalizable to the H-atom too. 

 Postulates of QM
 \begin{itemize}
     \item State described by an infite dimensoonal vector \( \psi(x,t) \) 
     \item All attributes to be specified in the \(|x> \)
     \item Normalization: Inner product is 1 or dirac delta. 
     \item Observables are replaced by hermitian operators. 
    
    \end{itemize}


\end{document}