\documentclass{article}
\usepackage{amsmath,latexsym,babel,amssymb,amsfonts,geometry}
\usepackage[dvipsnames]{xcolor}

\title{QM3 }
\author{Pushkar Mohile}
%New Commands
\newcommand{\note}[1]{\color{blue} Note: #1}
\newcommand{\set}[1]{ \left\{ #1 \right\} }
\newcommand{\olsi}[1]{\,\overline{\!{#1}}} % overline short italic
\newcommand{\bx}{\olsi{x}}
\begin{document}
\maketitle

\section{Tensors: Introduction}
Tensors are used to describe how quantities change on coordinate transforms. 
%
Consider a vector(?) space mapped by the coordinate system \( \set{x^1,x^2...}\) that smoothly maps 
to a new set of coordinates  \(\set{\bx^1, \bx^2..}\). 
\\
If both spaces are \(N\) dimensional, we can write 
\begin{align*}
    dx^i &= \sum_{\alpha = 1}^N\frac{\partial x^i}{\partial \bx^\alpha} d\bx^\alpha \\
    d\bx^i &= \sum_{\alpha = 1}^N\frac{\partial \bx^i}{\partial x^\alpha} dx^\alpha
\end{align*}
In the Einstein summation notation we can skip the \(\sum\) sign and sum over all possible values of
 repeated (dummy) indices. All the free indices give us a tensor quantity. For example
\begin{align*}
    \frac{dx^i}{dx^j} &= \delta_j^i \\
    \frac{d\bx^\alpha}{d\bx^\beta} &= \delta_\beta^\alpha
\end{align*}
Where \(\delta \) is the usual kronecker delta tensor. \\
The way mathematical objects 'transform' is used to classify them: \\
\textbf{Rank 1 contravariant tensor} \\
A group of \(N\) functions that depend on coordinates \(A^{i}\) is a contravariant tensor if it transforms on changing coordinates as
\begin{equation*}
    \olsi{A}^\alpha = \frac{\partial \bx^\alpha }{\partial x^j}A^j
\end{equation*} 
Observe that on multiplying by \(\frac{\partial x^i}{\partial\bx^\alpha} \) and summing over all \(\alpha\) \
we get 
\begin{align*}
    \frac{\partial x^i}{\partial \bx^\alpha} \olsi{A}^\alpha &=  \frac{\partial x^i}{\partial \bx^\alpha} \frac{\partial \bx^\alpha }{\partial x^j}A^j \\
    \frac{\partial x^i}{\partial \bx^\alpha} \olsi{A}^\alpha &= \delta_j^i A^j \\
    \frac{\partial x^i}{\partial \bx^\alpha} \olsi{A}^\alpha &= A^i
\end{align*}
Using the chain rule expansion to show \(\frac{\partial x^i}{\partial x^j} =  \frac{\partial x^i}{\partial \bx^\alpha} \frac{\partial \bx^\alpha }{\partial x^j} \).
This shows us the transformation is consistent for both coordinate changes \(\set{x^1,x^2..}\to \set{\bx^1,\bx^2..} \) and vice versa. \\
\textbf{Rank 1 covariant tensor} \\
A group of \(N\) functions that depend on coordinates \(A_{i}\) is a covariant tensor if it transforms on changing coordinates as
\begin{align*}
    \olsi{A}_\alpha &= \frac{\partial x^j }{\partial \bx^\alpha}A_j \\
    A_i &= \frac{\partial \bx^\alpha }{\partial x^i}\olsi{A}_\alpha
\end{align*} 
The gradient of a scalar field for example is a covariant vector. Let \(\phi(\set{x^i}) = \phi(\set{\bx^i}) \) 
be a scalar field that is left unchanged on transforming to a new coordinate system. It's gradient is 
given by a covariant tensor. 
\begin{align*}
    A_i &= \frac{\partial \phi}{\partial x^i}
\end{align*}
It is easy to observe on transforming coordinates, using the chain rule that  
\begin{align*}
    \olsi{A}_\alpha &= \frac{\partial \phi}{\partial \bx^i} \\
    \olsi{A}_\alpha &= \frac{\partial x^i}{\partial \bx^\alpha}\frac{\partial \phi}{\partial x^i} \\
    \olsi{A}_\alpha &= \frac{\partial x^i}{\partial \bx^\alpha}A_i
\end{align*}
Thus verifying that it is a covariant tensor. 
\\
\textbf{General tensor} \\
\(A\) is a general mixed tensor if it transforms as follows: 
\begin{equation*}
    \olsi{A}_{\beta_1\beta_2...\beta_q}^{\alpha_1\alpha_2...\alpha_p} = \frac{\partial\bx^{\alpha_1}}{\partial x^{i_1}}\frac{\partial\bx^{\alpha_2}}{\partial x^{i_2}}...\frac{\partial x^{j_1}}{\partial \bx^{\beta_1}}...A_{j_1j_2...j_q}^{i_1i_2...i_p}
\end{equation*}

2 tensors can only be added or susbtracted if they are of the same type ie they both 
have the same no of co and contravaraint indices. \\
%\textbf{Operations on tensors}
%We have the outer and inner product on 2 tensors that give us a new tensor. Tensor contraction reduces the
Index lowering and raising 


\section{Klein Gordan Equation L2}
Relativistic physics describes particle 
behaivour at high energies. We want our description of the laws of physics to be invariant under lorentz transformations. This means that the time coordinate should be treated similar to the spatial coordinates. \\
Further we want to continue to maintain the uncertainty principle in QM. \(\Delta x ~ \frac{h}{2\pi m_0c}\). If the particles come too close together you have interactions leading to pair production for \(E > 2m_0c^2\) \\
In relativistic QM we drop the assumption of conservation of particle number because of pair creation/annhilation \\

Everything is assumed to be 0 at spacetime infinity
\note{Look up time energy uncertainty in QM} \\
Total energy is 0


\end{document}